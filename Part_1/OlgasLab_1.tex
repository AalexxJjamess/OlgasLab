\documentclass[12pt]{article}

\usepackage{amsmath}
\usepackage{siunitx}

\title{Measuring Modes on Strings: Part 1}
\author{Alex Booth}

\begin{document}
    \maketitle

    \section{Introduction}
        This report details an investigation into the phenomena observed on a vibrating string.
        Several experiments took place, using a guitar, as guitars feature strings under tension that can freely vibrate.
        The frequency and amplitude of vibrations can be easily and precisely adjusted on a well-intonated guitar by fretting the strings along the fingerboard.
        Results are tabulated and plotted on graphs using MATLAB. 
    
    \section{Theory}
        A string vibrates with a fundamental, or in music theory tonic, frequency.
        This fundamental frequency is related to the length $L$ of a vibrating string:
        \begin{equation}
            f_1 = \frac{c}{2 L}
        \end{equation}
        The mass of a string and its length can be used to find a mass per unit length $M$.
        This mass per unit length can then be formulated with it's tension $T$ to find the speed of a transverse wave on a string:
        \begin{equation}\label{speed}
            c = \sqrt{\frac{T}{M}}
        \end{equation}
        Equation \ref{speed} allows us to formulate:
        \begin{equation}
            f_1 = \frac{\sqrt{\frac{T}{M}}}{2 L}
        \end{equation}
        If the ends of a string under tension are fixed, the fundamental frequency will have a wavelength equal to half the string's length.
        The frequencies of higher harmonics of the fundamental frequency are integer multiples of the fundamental frequency:
        \begin{equation}\label{harm}
            f_n = \frac{c}{2 L} n
        \end{equation}

    \section{Experiment One}
        \subsection{Methodology}
            In experiment one, the relationship between resonant frequency and string length were investigated.
            Preliminary measurements of mass and length were taken in order to find a mass per unit length $M = 0.012618\si{\kilogram\per\meter}$.
            Then the low E string was played in an open position, thus at it's maximum string length and at a frequency of $83\si{\hertz}$.
            Using equation \ref{harm} the higher harmonics $n = 1, 2 ,3 ...$ are predicted for the vibrating open E string.
            The sound of the vibrating string was captured using a microphone and passed into a Simulink frequency spectrum analyser.
            From this spectrum analyser the frequencies of the fundamental tone and it's higher harmonics could be ascertained.
            However, not every frequency in the harmonic series had a sufficient amplitude to be declared a peak by the Simulink spectrometer, meaning that the recorded harmonic series will have gaps compared to the array of predicted harmonic frequencies.
            Rearranging equation \ref{harm} allowed a wave speed $c_n$ to be found for each harmonic number, for both the predicted and recorded harmonics.
            To compare the error between the predicted and observed harmonic wave speeds, a mean wave speed was taken for the observed values.
            The difference between this mean wave speed $c$ and each predicted harmonic wave speed $c_n$ was then taken.
            These differences were then summed and divided by the number of harmonics to give a mean difference and thus average error:
            \begin{equation}
                \Delta c = \frac{1}{N}\sum_{n=1}^{N} | c_n - c |
            \end{equation}
            The array of predicted wave speeds for harmonic numbers up to harmonic number $n=6$ were tabluated:
            

           

        

\end{document}